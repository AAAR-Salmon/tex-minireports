\documentclass[a4paper, lualatex, ja=standard]{bxjsarticle}

\usepackage{luatexja}
\usepackage{amsmath, amssymb}
\usepackage{physics}
\usepackage{enumitem}
\usepackage{tcolorbox}

\newtcolorbox{dfntcb}{colframe=blue!50!black, colback=white}

\setpagelayout*{margin=20truemm}

\title{人工知能 第12回課題}
\author{氏名:  \\ 学生番号: }

\DeclareMathOperator*{\argmax}{arg\,max}

\begin{document}
\maketitle

隠れマルコフモデル(HMM)$\mathcal{M}$ は3つのパラメータ $(A,B,\pi)$ を持ち,
これを $\mathcal{M}=(A,B,\pi)$ と表記することとする.
ここで,$A$ は状態遷移確率分布,$B$ は観測シンボル分布,
$\pi$ は初期状態確率分布を表す.
また,モデル $\mathcal{M}$ は最大で $N$ 個の内部状態を持つとする.

HMMでは主に次の3つの問題を扱う.

\begin{enumerate}
\item
  出力シンボル系列 $O$ とモデル $\mathcal{M}$ から,
  $O$ を観測する尤度 $P(O|\mathcal{M})$ を求める

\item
  学習用シンボル $O$ を与えて尤度を最大にするモデルパラメータ
  $\mathcal{M}=(A,B,\pi)$ を求める

\item
  モデル $\mathcal{M}$ から出力シンボル系列 $O$ の出力を観測するとき,
  最も可能性が高い状態(遷移)系列 $S=[s_{1},\dots,s_{T}]$ を推定し,
  その系列の尤度を求める
\end{enumerate}

\section{ある出力シンボル系列 $O$ を観測する尤度を求める}

出力シンボル系列 $O$ の前側,または後ろ側から動的計画法を用いて計算する.

\subsection{Forward Procedure}

Forward変数 $\alpha_{t}(s)$ を次のように定義する.
\begin{dfntcb}
$\alpha_{t}(s)$: 時刻 $t$ までに出力系列 $[o_{1},\dots,o_{t}]$ が観測され,
時刻 $t$ での状態が $s$ である確率($0 \leq t \leq T$)
\end{dfntcb}

このとき,
\begin{align*}
\alpha_{0}(s) &= \pi_{s} \\
\alpha_{t}(s') &= \sum_{s}\alpha_{t-1}(s)a_{ss'}b_{ss'}(o_{t})
\end{align*}
として計算することで,
尤度 $\sum_{s}\alpha_{T}(s)$ を
時間計算量 $\mathcal{O}(N^{2} T)$ で計算することができる.

\subsection{Backward Procedure}

Backward変数 $\beta_{t}(s)$ を次のように定義する.
\begin{dfntcb}
$\beta_{t}(s)$: 時刻 $t$ よりあとに出力系列 $[o_{t+1},\dots,o_{T}]$ が観測され,
時刻 $t$ での状態が $s$ である確率($0 \leq t \leq T$)
\end{dfntcb}

このとき,
\begin{align*}
\beta_{T}(s) &= 1 \\
\beta_{t}(s') &= \sum_{s}\beta_{t+1}(s)a_{s's}b_{s's}(o_{t})
\end{align*}
として計算することで,
尤度 $\sum_{s}\pi_{s}\beta_{0}(s)$ を
時間計算量 $\mathcal{O}(N^{2} T)$ で計算することができる.

\section{Baum-Welch法: 尤度最大のモデルパラメータを求める}

2つの変数 $\xi_{t}(s,s'), \gamma_{t}(s)$ を導入する.
\begin{dfntcb}
$\xi_{t}(s,s')$: 観測系列が $O=[o_{1},\dots,o_{T}]$ であったとき,
時刻 $t$ に状態 $s$ から $s'$ へ遷移する確率

$\gamma_{t}(s)$: 時刻 $t$ のときの状態が $s$ である確率
\end{dfntcb}

このとき,
\begin{align*}
\xi_{t}(s,s') &= P(s_{t-1}=s,s_{t}=s'|O,\mathcal{M}) \\
&= \frac{P(s_{t-1}=s,s_{t}=s',O|\mathcal{M})}{P(O|\mathcal{M})} \\
&= \frac{\alpha_{t-1}(s)a_{ss'}b_{ss'}o_{t}\beta_{t}(s')}
        {\sum_{s,s'}\alpha_{t-1}(s)a_{ss'}b_{ss'}o_{t}\beta_{t}(s')} \\
\gamma_{t}(s) &= \sum_{s'}\xi_{t}(s,s')
\end{align*}
となる.

これらを利用し,適当なモデル $\mathcal{M}=(A,B,\pi)$ を仮定することで,
次のようにしてより出力系列の尤度の高くなるモデル
$\mathcal{M'}=(A',B',\pi')$ を推定することができる.
\begin{align*}
\pi'_{s} &=\gamma_{0}(s)=\frac{\alpha_{0}(s)\beta_{0}(s')}{P(O|\mathcal{M})} \\
a'_{ss'} &=\frac{\sum_{t}\xi_{t}(s,s')}{\sum_{t}\gamma_{t-1}(s)} \\
b'_{ss'}(o_{t}) &=\frac{\sum_{t|o_t}\xi_{t}(s,s')}{\sum_{t}\xi_{t}(s,s')}
\end{align*}

\section{Viterbiアルゴリズム: バックトラックによる最尤系列の探索}

2つの変数 $\delta_{t}(s), \psi_{t}(s)$ を導入する.
\begin{dfntcb}
$\delta_{t}(s)$: 時刻 $t$ で状態 $s$ にあるときに状態系列
$[s_{0},\dots,s_{t}(=s)]$ を取る確率の最大値

$\psi_{t}(s)$: 時刻 $t$ で状態 $s$ に遷移したときの直前の状態のうち
その遷移が状態系列の発生確率の最大値を取る状態
\end{dfntcb}

このとき,
\begin{align*}
\delta_{0}(s) &=\pi_{s} \\
\delta_{t}(s') &= \max_{s}\delta_{t-1}(s)a_{ss'}b_{ss'}(o_{t}) \\
\psi_{t}(s') &=\argmax_{s}\delta_{t-1}(s)a_{ss'}b_{ss'}(o_{t})
\end{align*}
とすると,$p^{*}=\max_{s}\delta_{T}(s)$ で最適な状態系列の生成確率を得る.
また,$s^{*}_{T}=\argmax_{s}\delta_{T}(s), s^{*}_{t}=\psi_{t+1}(s^{*}_{t+1})$ とすると,
$[s^{*}_{0},\dots,s^{*}_{T}]$ で最適な状態系列を得る.
\end{document}
