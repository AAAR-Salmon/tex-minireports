\documentclass[a4paper, lualatex, ja=standard]{bxjsarticle}

\usepackage{luatexja}
\usepackage{amsmath, amssymb}
\usepackage{physics}
\usepackage{bm}
\usepackage{enumitem}
\usepackage{tcolorbox}

\newtcolorbox{dfntcb}{colframe=blue!50!black, colback=white}

\setpagelayout*{margin=20truemm}

\title{人工知能 第13回課題}
\author{氏名:  \\ 学生番号: }

\DeclareMathOperator*{\argmax}{arg\,max}

\begin{document}
\maketitle

\section{誤差逆伝播法による学習}

\begin{enumerate}
\item 入力パターンベクトルと,対応する教師信号の組の集合を準備
\item シナプス結合荷重の初期値をランダムに設定
\item 入力パターンベクトルを1つ選び,各層のニューロンの出力を計算
\item 出力誤差 $J$ を最小化するように各層 $l$ の荷重 ${w'}_{ij}^{(l)}$ を決定し,学習率 $\alpha$ に基づき更新
  \begin{align*}
    J &= \frac{1}{2}\norm{\bm{o}^{(l)}-\bm{t}}^2 \\
    {w'}_{ij}^{(l)} &\leftarrow w_{ij}^{(l)}+\alpha\delta_{j}^{(l)}o_{i}^{(l-1)}
        \quad(\delta: \text{誤差信号})
  \end{align*}
\item 3に戻り,$w$ が収束するまで再更新
\end{enumerate}

\end{document}
