\documentclass[a4paper]{bxjsarticle}

\usepackage{luatexja}
\usepackage{luacode}
\usepackage{graphicx}

\usepackage{tikz}
\usetikzlibrary{calc,shapes}

\usepackage{enumitem}

\usepackage{adjustbox}
\usepackage{amsmath, amssymb}

\setpagelayout*{margin=10truemm}

% タイトルを調節
\makeatletter
\def\@maketitle{%
  \newpage\null
  \begin{center}%
    \let\footnote\thanks
    {\Large \@title \par}%
    \ifx\bxjs@subtitle\@undefined\else
      \vskip3\p@?
      {\normalsize \bxjs@subtitle\par}
    \fi
    \vskip 1em
    {\large
      \lineskip .5em
      \begin{tabular}[t]{c}%
        \@author
      \end{tabular}\par}%
    \vskip 1em
    {\large \@date}%
  \end{center}%
  \par\vskip 1.5em
  \ifvoid\@abstractbox\else\centerline{\box\@abstractbox}\vskip1.5em\fi
}
\makeatother

\title{2021年度 インターネットと認証連携 レポート課題1}
\author{\directlua{tex.print(os.getenv("AUTHOR_NAME"))}}
\date{提出日:2022年01月31日}


\begin{document}
\maketitle

図\ref{fig:init}を初期状態とするネットワークでの,PC1からPC2へのパケットの伝送の過程を説明する.
\begin{figure}[h]
  \begin{center}
    \begin{tikzpicture}
      \draw
      (0,0) node[draw, rectangle] (PC1) {PC1}
      (5,0) node[draw, ellipse] (RT1) {ルータ1}
      (10,0) node[draw, ellipse] (RT2) {ルータ2}
      (15,0) node[draw, rectangle] (PC2) {PC2};
      \draw[color=blue, thick]
      (PC1) -- (RT1) -- (RT2) -- (PC2);
      \draw[color=gray, dashed]
      ($(PC1)!0.5!(RT1)$) circle[x radius=2.3, y radius=1]
      +(0,-1) node[anchor=north]{10.1.0.0/16}
      ($(RT1)!0.5!(RT2)$) circle[x radius=2.3, y radius=1]
      +(0,-1) node[anchor=north]{10.0.0.0/16}
        ($(RT2)!0.5!(PC2)$) circle[x radius=2.3, y radius=1]
        +(0,-1) node[anchor=north]{10.2.0.0/16};
        \coordinate (yport) at (0,0.3);
        \draw
        (PC1.north |- yport) node[anchor=south west]{\tiny\begin{tabular}{l}
          1 \\ 10.1.0.100 \\ 12:34:00:00:00:01
        \end{tabular}}
        (RT1.north |- yport) node[anchor=south east]{\tiny\begin{tabular}{l}
          1 \\ 10.1.0.1 \\ 56:78:00:00:00:01
        \end{tabular}}
        (RT1.north |- yport) node[anchor=south west]{\tiny\begin{tabular}{l}
          2 \\ 10.0.0.1 \\ 56:78:00:00:00:02
        \end{tabular}}
        (RT2.north |- yport) node[anchor=south east]{\tiny\begin{tabular}{l}
          1 \\ 10.0.0.2 \\ 56:78:00:00:00:03
        \end{tabular}}
        (RT2.north |- yport) node[anchor=south west]{\tiny\begin{tabular}{l}
          2 \\ 10.2.0.1 \\ 56:78:00:00:00:04
        \end{tabular}}
        (PC2.north |- yport) node[anchor=south east]{\tiny\begin{tabular}{l}
          1 \\ 10.2.0.100 \\ 12:34:00:00:00:02
        \end{tabular}};
      \coordinate (yroot) at (0,1.3);
      \draw
      (PC1.north |- yroot) node[anchor=south]{\tiny\fbox{\begin{tabular}{lll}
        10.1.0.0/16 & Direct & 1 \\
          Default & 10.1.0.1 & 1
        \end{tabular}}}
        (RT1.north |- yroot) node[anchor=south]{\tiny\fbox{\begin{tabular}{lll}
          10.0.0.0/16 & Direct & 2 \\
          10.1.0.0/16 & Direct & 1 \\
          10.2.0.0/16 & 10.0.0.2 & 2 \\
          Default & 10.0.0.3 & 3
        \end{tabular}}}
        (RT2.north |- yroot) node[anchor=south]{\tiny\fbox{\begin{tabular}{lll}
          10.0.0.0/16 & Direct & 1 \\
          10.1.0.0/16 & 10.0.0.1 & 1 \\
          10.2.0.0/16 & Direct & 2 \\
          Default & 10.0.0.3 & 3
        \end{tabular}}}
        (PC2.north |- yroot) node[anchor=south]{\tiny\fbox{\begin{tabular}{lll}
          10.2.0.0/16 & Direct & 1 \\
          Default & 10.2.0.1 & 1
        \end{tabular}}};
    \end{tikzpicture}
  \end{center}
  \caption{ネットワーク構成(初期状態)}
  \label{fig:init}
\end{figure}

\begin{enumerate}
\item \relax[PC1]
  \begin{enumerate}
  \item
    送信元IPアドレスが10.1.0.100(PC1),宛先IPアドレスが10.2.0.100(PC2)のIPパケットを作成する.
  \item
    PC1のルーティングテーブルにおいて,
    宛先IPアドレス10.2.0.100にマッチする行はデフォルトルートのみであるため,デフォルトルートを選択する.
    よって次の宛先のIPアドレスは10.1.0.1であり,インタフェースは1番と分かる.
  \item
    IPアドレス10.1.0.1に対応するMACアドレスは56:78:00:00:00:01(ルータ1)であるから,
    送信元MACアドレスが12:34:00:00:00:01(PC1),
    宛先MACアドレスが56:78:00:00:00:01(ルータ1)のMACフレームを作成する.
  \item
    作成したMACフレームをインタフェース1から送信する.
  \end{enumerate}
\item \relax[ルータ1]
  \begin{enumerate}
  \item
    PC1から送られたMACフレームを受け取り,IPパケットを取り出す.
  \item
    ルータ1のルーティングテーブルにおいて,
    宛先IPアドレス10.2.0.100が10.2.0.0/16をキーとする行にマッチするため,その行を選択する.
    よって次の宛先のIPアドレスは10.0.0.2であり,インタフェースは2番と分かる.
  \item
    IPアドレス10.0.0.2に対応するMACアドレスは56:78:00:00:00:03(ルータ2)であるから,
    送信元MACアドレスが56:78:00:00:00:02(ルータ1),
    宛先MACアドレスが56:78:00:00:00:03(ルータ2)のMACフレームを作成する.
  \item
    作成したMACフレームをインタフェース2から送信する.
  \end{enumerate}
\item \relax[ルータ2]
  \begin{enumerate}
  \item
    ルータ1から送られたMACフレームを受け取り,IPパケットを取り出す.
  \item
    ルータ2のルーティングテーブルにおいて,
    宛先IPアドレス10.2.0.100が10.2.0.0/16をキーとする行にマッチするため,その行を選択する.
    よって次の宛先のIPアドレスは10.2.0.100であり,インタフェースは2番と分かる.
  \item
    IPアドレス10.2.0.100に対応するMACアドレスは12:34:00:00:00:02(PC2)であるから,
    送信元MACアドレスが56:78:00:00:00:04(ルータ2),
    宛先MACアドレスが12:34:00:00:00:02(PC2)のMACフレームを作成する.
  \item
    作成したMACフレームをインタフェース2から送信する.
  \end{enumerate}
\item \relax[PC2]
  \begin{enumerate}
  \item
    ルータ2から送られたMACフレームを受け取り,IPパケットを取り出す.
  \item
    宛先IPアドレスと自身のIPアドレスが一致するため,伝送を終了する.
  \end{enumerate}
\end{enumerate}
\end{document}
