\documentclass[a4paper]{bxjsarticle}

\usepackage{luatexja}
\usepackage{luacode}

\usepackage{enumitem}

\usepackage{fancyvrb}
\usepackage{adjustbox}
\usepackage{amsmath, amssymb}
\usepackage{mathtools}
\usepackage{physics}

\setpagelayout*{margin=20truemm}

\title{2021年度 応用解析 レポート課題}
\author{\directlua{tex.print(os.getenv("AUTHOR_NAME"))}}
\date{提出日:2021年8月9日}


\begin{document}
	\maketitle


	\section{概要}
	定積分値の計算を台形積分,シンプソン積分およびロンバーグ積分で計算を行い,これらの誤差について評価する.


	\section{実験内容}
	\ref{sec:program}に示すCのプログラムを用いて台形積分とシンプソン積分,ロンバーグ積分を
	\begin{equation}
		I=\int_{1}^{2}\qty(\frac{1}{x^3}+\frac{2}{x^5})dx=\frac{27}{32}=0.84375
	\end{equation}
	について行う.

	台形積分とシンプソン積分については,それぞれ分割数が$n=2,4,8,16$の4通りの場合について計算した.

	また,ロンバーグ積分は,分割数$n$の台形積分を$I^{(0)}(n)$とし,
	加速$I^{(k+1)}(n)=\dfrac{4^k I^{(k)}(n)-I^{(k)}(n/2)}{4^k-1}$を利用して,
	$I^{(0)}(n)(n=1,2,4,8,16)$の値から$I^{(4)}(16)$を求める.


	\section{計算結果}
	実行時の出力はそれぞれ以下のようになった.

	\VerbatimInput[fontsize=\small, xleftmargin=10mm, frame=lines]{trap_simpson_error.csv}
	\VerbatimInput[fontsize=\small, xleftmargin=10mm, frame=lines]{romberg_error.csv}

	\noindent
	これを表にまとめると表\ref{tab:trap_simpson},表\ref{tab:romberg}のようになる.

	\begin{table}[bt]
		\caption{台形積分とシンプソン積分の誤差}
		\label{tab:trap_simpson}
		\begin{center}
			\begin{tabular}{r|cc|cc}\hline
				$n$ & 台形積分 & 誤差 & シンプソン積分 & 誤差 \\ \hline
				2 & 1.076710390947 & 0.232960390947 & 0.904363854595 & 0.060613854595 \\
				4 & 0.907305923146 & 0.063555923146 & 0.850837767213 & 0.007087767213 \\
				8 & 0.860071570485 & 0.016321570485 & 0.844326786265 & 0.000576786265 \\
				16 & 0.847859818988 & 0.004109818988 & 0.843789235155 & 0.000039235155 \\ \hline
			\end{tabular}
		\end{center}
	\end{table}

	\begin{table}
		\caption{ロンバーグ積分の誤差}
		\label{tab:romberg}
		\begin{center}
			\begin{tabular}{r|cc|cc|cc} \hline
				$n$ & $I^{(0)}(n)$ & 誤差 & $I^{(1)}(n)$ & 誤差 & $I^{(2)}(n)$ & 誤差 \\ \hline
				 1 & 1.593750000000 & $7.50e{-1}$ &&&& \\
				 2 & 1.076710390947 & $2.33e{-1}$ & 0.904363854595 & $6.06e{-2}$ && \\
				 4 & 0.907305923146 & $6.36e{-2}$ & 0.850837767213 & $7.09e{-3}$ & 0.847269361387 & $3.52e{-3}$ \\
				 8 & 0.860071570485 & $1.63e{-2}$ & 0.844326786265 & $5.77e{-4}$ & 0.843892720869 & $1.43e{-4}$ \\
				16 & 0.847859818988 & $4.11e{-3}$ & 0.843789235155 & $3.92e{-5}$ & 0.843753398414 & $3.40e{-6}$ \\ \hline
			\end{tabular}
			\begin{tabular}{r|cc|cc} \hline
				$n$ & $I^{(3)}(n)$ & 誤差 & $I^{(4)}(n)$ & 誤差 \\ \hline
				 8 & 0.843839123400 & $8.91e{-5}$ && \\
				16 & 0.843751186947 & $1.19e{-6}$ & 0.843750842098 & $8.42e{-7}$ \\ \hline
			\end{tabular}
		\end{center}
	\end{table}


	\section{考察}
	台形積分とシンプソン積分について,分割数$n$と誤差の両対数グラフを取ると図\ref{fig:trap_simpson_error}のようになる.

	\begin{figure}[bt]
		\begin{center}
			\includegraphics{trap_simpson_error.pdf}
		\end{center}
		\caption{台形積分とシンプソン積分の誤差のグラフ}
		\label{fig:trap_simpson_error}
	\end{figure}

	両対数グラフで傾きが負の直線を描くことから,誤差は分割幅$h=1/n$の多項式となることがわかる.傾きは台形積分とシンプソン積分でそれぞれ$-1.94\dots, -3.53\dots$であり,およそ$O(h^2)$,$O(h^4)$で評価できる.

	また,ロンバーグ積分について,加速回数$k$に対し,分割数$n$と誤差の両対数グラフを取ると図\ref{fig:romberg_error}のようになる.

	\begin{figure}[bt]
		\begin{center}
			\includegraphics{romberg_error.pdf}
		\end{center}
		\caption{ロンバーグ積分の誤差のグラフ}
		\label{fig:romberg_error}
	\end{figure}

	同様にこのグラフの傾きを求めると,$k=0,1,2,3$のとき$-1.87\dots, -3.53\dots, -5.00\dots, -6.23\dots$となっており,帰納的に$O(h^{2+1.5k})$程度であると見積もることができる.


	\appendix
	\section{使用したプログラム} \label{sec:program}
	\subsection{台形積分とシンプソン積分}
	\VerbatimInput[fontsize=\small, tabsize=4, frame=lines, numbers=left, xleftmargin=10mm]{trap_simpson.c}

	\subsection{ロンバーグ積分}
	\VerbatimInput[fontsize=\small, tabsize=2, frame=lines, numbers=left, xleftmargin=10mm]{romberg.c}
\end{document}