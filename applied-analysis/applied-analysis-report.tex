\documentclass[a4paper]{bxjsarticle}

\usepackage{luatexja}
\usepackage{luacode}

\usepackage{enumitem}

\usepackage{fancyvrb}
\usepackage{adjustbox}
\usepackage{amsmath, amssymb}
\usepackage{mathtools}
\usepackage{physics}
% \usepackage{tikz}
% \usepackage[RPvoltages]{circuitikz}
% \usepackage{diagbox}
% \usepackage{karnaugh-map}

% \usetikzlibrary{calc}

\setpagelayout*{margin=20truemm}

\title{2021年度 応用解析 レポート課題}
\author{\directlua{tex.print(os.getenv("AUTHOR_NAME"))}}
\date{提出日:2021年8月9日}


\begin{document}
	\maketitle


	\section{概要}
	定積分値の計算を台形積分,シンプソン積分およびロンバーグ積分で計算を行い,これらの誤差について評価する.


	\section{実験内容}
	\ref{sec:program}に示すCのプログラムを用いて台形積分とシンプソン積分,ロンバーグ積分を
	\begin{equation}
		I=\int_{1}^{2}\qty(\frac{1}{x^3}+\frac{2}{x^5})dx=\frac{27}{32}=0.84375
	\end{equation}
	について行う.

	台形積分とシンプソン積分については,それぞれ分割数が$n=2,4,8,16$の4通りの場合について計算した.

	また,ロンバーグ積分は,分割数$n$の台形積分を$I^{(0)}(n)$とし,
	加速$I^{(k+1)}(n)=\dfrac{4^k I^{(k)}(n)-I^{(k)}(n/2)}{4^k-1}$を利用して,
	$I^{(0)}(n)(n=1,2,4,8,16)$の値から$I^{(4)}(16)$を求める.


	\section{計算結果}
	実行時の出力はそれぞれ以下のようになった.

	\begin{Verbatim}[fontsize=\small, xleftmargin=10mm, gobble=2, frame=lines]
		 2, 1.076710390947, 0.232960390947, 0.904363854595, 0.060613854595
		 4, 0.907305923146, 0.063555923146, 0.850837767213, 0.007087767213
		 8, 0.860071570485, 0.016321570485, 0.844326786265, 0.000576786265
		16, 0.847859818988, 0.004109818988, 0.843789235155, 0.000039235155
	\end{Verbatim}
	\begin{Verbatim}[fontsize=\small, xleftmargin=10mm, gobble=2, frame=lines]
		0,  1, 1.593750000000, 0.750000000000
		0,  2, 1.076710390947, 0.232960390947
		0,  4, 0.907305923146, 0.063555923146
		0,  8, 0.860071570485, 0.016321570485
		0, 16, 0.847859818988, 0.004109818988
		1,  2, 0.904363854595, 0.060613854595
		1,  4, 0.850837767213, 0.007087767213
		1,  8, 0.844326786265, 0.000576786265
		1, 16, 0.843789235155, 0.000039235155
		2,  4, 0.847269361387, 0.003519361387
		2,  8, 0.843892720869, 0.000142720869
		2, 16, 0.843753398414, 0.000003398414
		3,  8, 0.843839123400, 0.000089123400
		3, 16, 0.843751186947, 0.000001186947
		4, 16, 0.843750842098, 0.000000842098
	\end{Verbatim}

	\noindent
	これを表にまとめると表\ref{tab:trap_simpson},表\ref{tab:romberg}のようになる.

	\begin{table}[bt]
		\caption{台形積分とシンプソン積分の誤差}
		\label{tab:trap_simpson}
		\begin{center}
			\begin{tabular}{r|cc|cc}\hline
				$n$ & 台形積分 & 誤差 & シンプソン積分 & 誤差 \\ \hline
				2 & 1.076710390947 & 0.232960390947 & 0.904363854595 & 0.060613854595 \\
				4 & 0.907305923146 & 0.063555923146 & 0.850837767213 & 0.007087767213 \\
				8 & 0.860071570485 & 0.016321570485 & 0.844326786265 & 0.000576786265 \\
				16 & 0.847859818988 & 0.004109818988 & 0.843789235155 & 0.000039235155 \\ \hline
			\end{tabular}
		\end{center}
	\end{table}

	\begin{table}
		\caption{ロンバーグ積分の誤差}
		\label{tab:romberg}
		\begin{center}
			\begin{tabular}{r|cc|cc|cc} \hline
				$n$ & $I^{(0)}(n)$ & 誤差 & $I^{(1)}(n)$ & 誤差 & $I^{(2)}(n)$ & 誤差 \\ \hline
				 1 & 1.593750000000 & $7.50e{-1}$ &&&& \\
				 2 & 1.076710390947 & $2.33e{-1}$ & 0.904363854595 & $6.06e{-2}$ && \\
				 4 & 0.907305923146 & $6.36e{-2}$ & 0.850837767213 & $7.09e{-3}$ & 0.847269361387 & $3.52e{-3}$ \\
				 8 & 0.860071570485 & $1.63e{-2}$ & 0.844326786265 & $5.77e{-4}$ & 0.843892720869 & $1.43e{-4}$ \\
				16 & 0.847859818988 & $4.11e{-3}$ & 0.843789235155 & $3.92e{-5}$ & 0.843753398414 & $3.40e{-6}$ \\ \hline
			\end{tabular}
			\begin{tabular}{r|cc|cc} \hline
				$n$ & $I^{(3)}(n)$ & 誤差 & $I^{(4)}(n)$ & 誤差 \\ \hline
				 8 & 0.843839123400 & $8.91e{-5}$ && \\
				16 & 0.843751186947 & $1.19e{-6}$ & 0.843750842098 & $8.42e{-7}$ \\ \hline
			\end{tabular}
		\end{center}
	\end{table}


	\section{考察}
	台形積分とシンプソン積分について,分割数$n$と誤差の両対数グラフを取ると図\ref{fig:trap_simpson_error}のようになる.

	\begin{figure}[bt]
		\begin{center}
			\includegraphics{trap_simpson_error.pdf}
		\end{center}
		\caption{台形積分とシンプソン積分の誤差のグラフ}
		\label{fig:trap_simpson_error}
	\end{figure}

	両対数グラフで傾きが負の直線を描くことから,誤差は分割幅$h=1/n$の多項式となることがわかる.傾きは台形積分とシンプソン積分でそれぞれ$-1.94\dots, -3.53\dots$であり,およそ$O(h^2)$,$O(h^4)$で評価できる.

	また,ロンバーグ積分について,加速回数$k$に対し,分割数$n$と誤差の両対数グラフを取ると図\ref{fig:romberg_error}のようになる.

	\begin{figure}[bt]
		\begin{center}
			\includegraphics{romberg_error.pdf}
		\end{center}
		\caption{ロンバーグ積分の誤差のグラフ}
		\label{fig:romberg_error}
	\end{figure}

	同様にこのグラフの傾きを求めると,$k=0,1,2,3$のとき$-1.87\dots, -3.53\dots, -5.00\dots, -6.23\dots$となっており,帰納的に$O(h^{2+1.5k})$程度であると見積もることができる.


	\appendix
	\section{使用したプログラム} \label{sec:program}
	\subsection{台形積分とシンプソン積分}
	\begin{Verbatim}[fontsize=\small, tabsize=4, frame=lines, numbers=left, xleftmargin=10mm, gobble=2]
		#include <stdio.h>

		#define I (27.0/32.0) // true value
		#define LOW 1.0
		#define HIGH 2.0
		#define MAXN 16

		// integrand
		double f(double x) {
			return 1.0 / (x * x * x) + 2.0 / (x * x * x * x * x);
		}

		int main(void) {
			double I1, I2, h, y[MAXN + 1];
			for (int n = 2; n <= MAXN; n *= 2) {
				I1 = I2 = 0.0;
				h = (HIGH - LOW) / n;
				for (int i = 0; i <= n; i++) {
					y[i] = f(LOW + h * i);
				}

				// trapezoidal
				I1 += (y[0] + y[n]) / 2.0;
				for (int i = 1; i < n; i++) {
					I1 += y[i];
				}
				I1 *= h;

				// simpson
				double _sum;
				I2 += (y[0] + y[n]);
				_sum = 0.0;
				for (int i = 1; i < n; i += 2) {
					_sum += y[i];
				}
				I2 += _sum * 4;
				_sum = 0.0;
				for (int i = 2; i < n; i += 2) {
					_sum += y[i];
				}
				I2 += _sum * 2;
				I2 *= h / 3.0;

				printf("%2d, %.12f, %.12f, %.12f, %.12f\n", n, I1, I1 - I, I2, I2 - I);
			}

			return 0;
		}
	\end{Verbatim}

	\subsection{ロンバーグ積分}
	\begin{Verbatim}[fontsize=\small, tabsize=2, frame=lines, numbers=left, xleftmargin=10mm, gobble=2]
		#include <stdio.h>

		#define I_TRUE (27.0/32.0) // true value
		#define LOW 1.0
		#define HIGH 2.0
		#define LOG_MAXN 4

		// integrand
		double f(double x) {
			return 1.0 / (x * x * x) + 2.0 / (x * x * x * x * x);
		}

		int main(void) {
			double I[5][5], y[1 << LOG_MAXN + 1];
			for (int i = 0; i <= LOG_MAXN; i++) {
				int n = 1 << i;
				double _I0 = 0.0;
				double h = (HIGH - LOW) / n;
				for (int i = 0; i <= n; i++) {
					y[i] = f(LOW + h * i);
				}

				// trapezoidal
				_I0 += (y[0] + y[n]) / 2.0;
				for (int i = 1; i < n; i++) {
					_I0 += y[i];
				}
				_I0 *= h;

				I[0][i] = _I0;
				printf("0, %2d, %.12f, %.12f\n", n, _I0, _I0 - I_TRUE);
			}

			double pow4 = 4;
			for (int k = 1; k <= LOG_MAXN; k++) {
				for (int log_n = k; log_n <= LOG_MAXN; log_n++) {
					I[k][log_n] = (pow4 * I[k - 1][log_n] - I[k - 1][log_n - 1]) / (pow4 - 1);
					printf("%1d, %2d, %.12f, %.12f\n", k, 1 << log_n, I[k][log_n], I[k][log_n] - I_TRUE);
				}
				pow4 *= 4;
			}


			return 0;
		}
	\end{Verbatim}
\end{document}