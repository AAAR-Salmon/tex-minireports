\documentclass[a4paper, lualatex, ja=standard]{bxjsarticle}

\usepackage{luatexja}
\usepackage{amsmath, amssymb}
\usepackage{enumitem}

\setpagelayout*{margin=20truemm}

\title{人工知能 第11回課題}
\author{氏名:  \\ 学生番号: }

\begin{document}
\maketitle

\section{資料演習問題3.2.3}

\verb|fun f(a:int, b, c, d, e) =|
\begin{enumerate}[label=\alph*)]
\item \verb|if a<b+c then d else e;|

  \verb|a| の型は \verb|int| であるから \verb|b+c| の型は \verb|int| であり,
  \verb|b|,\verb|c| の型は\verb|int|.

  if式のthen句とelse句は同じ型を返さなければならないから,
  \verb|d|,\verb|e| の型は \verb|'a|.

\item \verb|if a<b then c else d;|

  \verb|a| の型は \verb|int| であるから \verb|b| の型は \verb|int|.

  if式のthen句とelse句は同じ型を返さなければならないから,
  \verb|c|,\verb|d| の型は \verb|'a|.

  \verb|e| の型は \verb|'b|.

\item \verb|if a<b then b+c else d+e;|

  \verb|a| の型は \verb|int| であるから \verb|b| の型は \verb|int|.

  \verb|b| の型は \verb|int| であるから \verb|b+c| の型は \verb|int| であり,
  \verb|c| の型は \verb|int|.

  then句 \verb|b+c| の型は \verb|int| であるから,
  else句 \verb|d+e| の型は \verb|int| であり,
  \verb|d|,\verb|e| の型は \verb|int|.

\item \verb|if a<b then b<c else d;|

  \verb|a| の型は \verb|int| であるから \verb|b| の型は \verb|int|.

  \verb|b| の型は \verb|int| であるから \verb|c| の型は \verb|int|.

  \verb|b<c| の型は \verb|bool| であるから \verb|d| の型は \verb|bool|.

  \verb|e| の型は \verb|'a|. 

\item \verb|if b<c then a else c+d;|

  \verb|a| の型は \verb|int| であるから \verb|c+d| の型は \verb|int| であり,
  \verb|c|,\verb|d| の型は \verb|int|.

  \verb|c| の型は \verb|int| であるから \verb|b| の型は \verb|int|.

  \verb|e| の型は \verb|'a|.

\item \verb|if b<c then d else e;|

  \verb|b<c| が含まれるから \verb|b|,\verb|c| の型は \verb|int|.

  \verb|d|,\verb|e| の型は等しく \verb|'a|.

\item \verb|if b<c then d+e else d*e;|

  \verb|b<c| が含まれるから \verb|b|,\verb|c| の型は \verb|int|.

  \verb|d<e| が含まれるから \verb|d|,\verb|e| の型は \verb|int|.
\end{enumerate}

\section{型推論に関する問題集 問3.3}

\begin{enumerate}[label=\arabic*., start=2]
\item \verb|fn x => twice id x|

  \verb|x| の型を \verb|'a| と仮定する.
  このとき,\verb|id| の型を \verb|'b -> 'b| とすると
  \verb|twice id| の型は\verb|'b -> 'b|.

  \verb|twice id x = (twice id) x| より,全体の型は \verb|'a -> 'a|.

\item \verb|fun thrice f x = f (f (f x))|

  \verb|f| は繰り返し適用できるからその型は \verb|'a -> 'a|.

  \verb|f x| が含まれるから \verb|x| の型は \verb|'a|.

  したがって,全体の型は \verb|('a -> 'a) -> 'a -> 'a|
\end{enumerate}

\section{型推論に関する問題集 問3.5}
\begin{enumerate}[label=\arabic*, start=3]
\item \verb|fun S x y z = (x z) (y z)|

  \verb|z| の型を \verb|'a| と仮定する.

  \verb|z| を \verb|y| に適用しているから,\verb|y| の型は \verb|'a -> 'b|.

  \verb|z| を \verb|x| に適用し,その関数に \verb|(y z)| が適用されているから,
  \verb|x| の型は \verb|'a -> 'b -> 'c|.

  したがって,\verb|S| の型は \verb|('a -> 'b -> 'c) -> ('a -> 'b) -> 'a -> 'c|.
\end{enumerate}

\end{document}