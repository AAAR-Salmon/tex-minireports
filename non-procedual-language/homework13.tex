\documentclass[a4paper, lualatex, ja=standard]{bxjsarticle}

\usepackage{luatexja}
\usepackage{amsmath, amssymb}
\usepackage{enumitem}

\setpagelayout*{margin=20truemm}

\title{非手続き型言語 第13回課題}
\author{氏名:  \\ 学生番号: }

\begin{document}
\maketitle

\section{型推論に関する問題集 問3.5}

\begin{enumerate}
\item[3.] \verb|fun A x y z = z y x|

  \verb|x| の型を \verb|'a| と仮定する.
  \verb|y| を \verb|z| に適用し,その関数に \verb|x| が適用されているから,
  \verb|y| の型は \verb|'b|,\verb|z| の型は \verb|'b -> 'a -> 'c|.

  したがって,全体の型は \verb|'a -> 'b -> ('b -> 'a -> 'c) -> c|.

\item[4.] \verb|fun B f g = f g g|

\verb|g| の型を \verb|'a| と仮定する.
\verb|g| を \verb|f| に適用し,その関数に \verb|g| が適用されているから,
\verb|f| の型は \verb|'a -> 'a -> 'b|.

したがって,全体の型は \verb|('a -> 'a -> 'b) -> 'a -> 'b|.
\end{enumerate}

\section{型推論に関する問題集 問3.7}

\begin{enumerate}
\item[2.] \verb|fun f x y z = x (y z) : int|

  \verb|x (y z) : int| は \verb|(x (y z)) : int| に等しい.

  ここで,\verb|z| の型を \verb|'b| と仮定する.
  \verb|z| を \verb|y| に適用しているから,\verb|y| の型は \verb|'b -> 'a|.
  \verb|(y z)| を \verb|x| に適用しているから,\verb|x| の型は \verb|'a -> int|.

  したがって,全体の型は \verb|('a -> int) -> ('b -> 'a) -> 'b -> int|.

\item[3.] \verb|fun f x y z = (x y z) : int|

  \verb|z| の型を \verb|'b| と仮定する.
  \verb|y| を \verb|x| に適用し,その関数に \verb|z| が適用されているから,
  \verb|y| の型は \verb|'a|,\verb|x| の型は \verb|'a -> 'b -> int|.

  したがって,全体の型は \verb|('a -> 'b -> int) -> 'a -> 'b -> int|.

\item[4.] \verb|fun f x y z = x y (z : int)|

  明らかに \verb|z| の型は \verb|int| である.

  \verb|y| を \verb|x| に適用し,その関数に \verb|z| が適用されているから,
  \verb|y| の型は \verb|'a|,\verb|x| の型は \verb|'a -> int -> 'b|.

  したがって,全体の型は \verb|('a -> int -> 'b) -> 'a -> int -> 'b|.

\item[5.] \verb|fun f x y z = x (y z : int)|

  \verb|z| の型を \verb|'b| と仮定する.
  \verb|z| を \verb|y| に適用しているから,\verb|y| の型は \verb|'b -> int|.
  \verb|y z| を \verb|x| に適用しているから,\verb|x| の型は \verb|int -> 'a|.

  したがって,全体の型は \verb|(int -> 'a) -> ('b -> int) -> 'b -> 'a|.
\end{enumerate}

\section{型推論に関する問題集 問3.11}

\begin{enumerate}
\item[1.] \verb|fn x => x > 1|

  \verb|x > 1| が含まれるから,\verb|x| の型は \verb|int|.

  全体の型は \verb|int -> bool|.

\item[2.] \verb|fn x => fn y => fn z => (x y, x "Ada", y > z)|

  \verb|"Ada"| を \verb|x| に適用しているから,\verb|x| の型は \verb|str -> 'a|.
  \verb|y| を \verb|x| に適用しているから,\verb|y| の型は \verb|str|.
  \verb|y > z| が含まれるから,\verb|z| の型は \verb|str|.

  全体の型は \verb|(str -> 'a) -> str -> str -> 'a * 'a * bool|.

\end{enumerate}

\section{資料演習問題5.6.4}

\begin{verbatim}
fun comp F G =
  let
    fun C x = G(F(x))
  in
    C
  end;

fun add1 x = x + 1;
\end{verbatim}
このとき,\verb|comp| の型は \verb|('a -> 'b) -> ('b -> 'c) -> 'a -> 'c|,
\verb|add1| の型は \verb|int -> int|.

\begin{enumerate}[label=\alph*)]
\item \verb|val compA1 = comp add1;|

  \verb|comp| の型は \verb|('a -> 'b) -> ('b -> 'c) -> 'a -> 'c| であるから,
  \verb|compA1| の型は \\ \verb|(int -> 'a) -> int -> 'a|.

\item \verb|val compCompA1 = comp compA1;|

  \verb|comp| の型は \verb|('a -> 'b) -> ('b -> 'c) -> 'a -> 'c| であるから,
  \verb|compCompA1| の型は \\ \verb|((int -> 'a) -> 'b) -> (int -> 'a) -> 'b|.

\item \verb|val f = compA1 add1;|

  \verb|compA1| の型は \verb|(int -> 'a) -> int -> 'a| であるから,
  \verb|f| の型は \verb|int -> int|.

\item \verb|f(2);|

\begin{verbatim}
f(2) = compA1 add1 2
     = comp add1 add1 2
     = add1 (add1 2)
     = add1 3
     = 4.
\end{verbatim}

\item \verb|val g = compCompA1 compA1;|

  \verb|compCompA1| の型は
  \verb|((int -> 'a) -> 'b) -> (int -> 'a) -> 'b|,\\
  \verb|compA1| の型は \verb|(int -> 'a) -> int -> 'a| であるから,
  \verb|g| の型は \verb|(int -> 'a) -> int -> 'a|.

\item \verb|val h = g add1;|

  \verb|g| の型は \verb|(int -> 'a) -> int -> 'a| であるから,
  \verb|h| の型は \verb|int -> int|.

\item \verb|h(2);|

\begin{verbatim}
h(2) = g add1 2
     = compCompA1 compA1 add1 2
     = comp compA1 compA1 add1 2
     = comp (comp add1) (comp add1) add1 2
     = comp add1 (comp add1 add1) 2
     = comp add1 add1 (add1 2)
     = add1 (add1 (add1 2))
     = add1 (add1 3)
     = add1 4
     = 5.
\end{verbatim}
\end{enumerate}

\end{document}